% !TEX root = ../main.tex
% !TEX spellcheck = en-US

\section{Proof that $\Fsrfe^{P}$ is Enough to Realize $\FFAKE^{P}$}
\label{sec:ygcfakeproof}

In Figure~\ref{fig:YGCFAKE}, we describe a protocol $\PfakeYGC$ which trivially realizes $\FFAKE^{P}$ in the $\Fsrfe^{P}$-hybrid model.
\leo[inline]{Out of curiousity (and to help the reader), would be correct to say  ``$\Fsrfe^{P}$ is essentially the same as $\FFAKE^{P}$, and any protocol that realizes one also realizes the other?''}

\julia[inline]{$\PfakeYGC$ has nothing to do with YGC, so the name is a bit misleading. If someone later references the theorem this might be confusing.}

\begin{figure}[htbp]
  \centering
  \begin{fboxenv}
    \begin{minipage}{0.95\textwidth}
    	Upon receiving the input $(\sid, \apw)$, $\aparty \in \{\firstparty, \secondparty\}$ does the following:
	\begin{itemize}
	\item Sends $(\Init, \sid)$ to $\Fsrfe^{P}$;
	\item Sends $(\NewSession, \sid, \apw)$ to $\Fsrfe^{P}$;
	\item Waits for $\key$ from $\Fsrfe^{P}$, and 
	\item Outputs $\key$.
	\end{itemize}
    \end{minipage}
  \end{fboxenv}
  \caption{Protocol $\PfakeYGC$ realizing $\FFAKE^{P}$ in the $\Fsrfe^{P}$-hybrid model.}
  \label{fig:YGCFAKE}
\end{figure}

\begin{theorem}
Protocol $\PfakeYGC$ realizes $\FFAKE^{P}$ in the $\Fsrfe^{P}$-hybrid model.
\end{theorem}

\expl{Canetti~\etal~\cite{PKC:CDVW12} make a similar claim but only defend it in a ``full" version that is not available online.}

\begin{proof}
For every efficient adversary $\AdvA$, we describe a simulator $\SimYGCFAKE$ in Figure~\ref{fig:SimulatorYGCFAKE} such that no efficient environment can distinguish an execution with the real protocol $\PfakeYGC$ and $\AdvA$ from an execution with the ideal functionality $\FFAKE^{P}$ and $\SimYGCFAKE$.
Since the environment does not get any information about the honest parties except their output, all the simulator needs to do is respond to queries to $\Fsrfe^{P}$.
%$\SimYGCFAKE$ controls the ideal functionality $\Fsrfe^{P}$, so it can see all the adversarial inputs to that ideal functionality, but the adversary cannot see honest parties' inputs or outputs.
Since the honest party does nothing except query the ideal functionality $\Fsrfe^{P}$, and its output gets replaced by values chosen by $\FFAKE^{P}$, there is nothing to simulate.

\begin{figure}[htbp]
  \centering
  \begin{fboxenv}
    \begin{minipage}{0.95\textwidth}
    $\SimYGCFAKE$ responds to queries to $\Fsrfe^{P}$ as follows:
	\begin{itemize}
	\item Upon getting $(\Init, \sid)$ from $\AdvA$ on behalf of corrupt party $\otherparty \in \{\firstparty, \secondparty\}$, $\SimYGCFAKE$ 
	does nothing.
	%executes the code in Figure~\ref{fig:func-SRFE}.
	\item Upon getting $(\Init, \sid, \aparty, H, \sid_{H})$ from $\AdvA$, $\SimYGCFAKE$ 
	does nothing.
	%executes the code in Figure~\ref{fig:func-SRFE}.
	\item Upon getting $(\NewSession, \sid, \apw)$ from $\AdvA$ on behalf of honest party $\aparty \in \{\firstparty, \secondparty\}$, $\SimYGCFAKE$ does nothing.
	\item Upon getting $(\NewSession, \sid, \otherpw')$ from $\AdvA$ on behalf of corrupt party $\otherparty \in \{\firstparty, \secondparty\}$, $\SimYGCFAKE$:
	\begin{itemize}
	%\item Executes the code in Figure~\ref{fig:func-SRFE};
	\item Records $\otherpw'$;
	\item Sends $(\TestPwd, \sid, \aparty, \otherpw')$ to $\FFAKE^{P}$;
	\item If $d(\apw, \otherpw') \leq \delta$, $\SimYGCFAKE$ learns $\apw$.
	\end{itemize}
	\item Upon getting a $(\TestPwd, \sid, \aparty)$ query from $\AdvA$, 
	$\SimYGCFAKE$ responds with the output of the $\TestPwd$ query above.
	\item Upon getting a $(\NewKey, \sid, \aparty, \akey)$ query from $\AdvA$, 
	if $\aparty$ is corrupt, $\SimYGCFAKE$ outputs $\akey$ to $\aparty$.
	In any case, $\SimYGCFAKE$ forwards $(\NewKey, \sid, \aparty, \akey)$ to $\FFAKE^{P}$.
	\end{itemize}
	Additionally, $\SimYGCFAKE$ forwards all other instructions from \Env to \AdvA and reports all output of \AdvA towards \Env. Instructions of corrupting a player are only obeyed if they are received before the protocol started, i.e., before \Sim received any \NewKey query from $\FFAKE^{P}$.
    \end{minipage}
  \end{fboxenv}
  \caption{Simulator $\SimYGCFAKE$ for $\PfakeYGC$.}
  \label{fig:SimulatorYGCFAKE}
\end{figure}

All that remains to show is that the values produced by $\FFAKE^{P}$ and by $\Fsrfe^{P}$ are identically distributed.
We describe the outputs of $\FFAKE^{P}$ and $\Fsrfe^{P}$ in Figure~\ref{fig:outputtables}.
The table enumerates all possible cases in the functionalities. 
Cases in $\Fsrfe^{P}$ are described in terms of the distances between \passwords: between the honest parties' \passwords if no man in the middle attack occurred, and between adversarial and honest \passwords if it did.
(If the adversary engaged one party but not the other, only one of the distances is filled in.)
Cases in $\FFAKE^{P}$ are described in terms of record markings (``fresh'', ``compromised'' or ``interrupted'').
There is a one-to-one mapping between the cases in $\Fsrfe^{P}$ and $\FFAKE^{P}$ such that the outputs for the parties, whether they are honest or corrupt, are identically distributed.
Those outputs are described in the last three columns of the table, as tuples of values the first of which is output to $\firstparty$, and the second of which is output to $\secondparty$.
$a, b$ are adversarially chosen values (which may or may not be distinct).
$r, s$ are independent, uniformly random values.

%``(g, g)'' denotes the case where both parties are good; ``(g, b)'' denotes the case where $\Party_i$ is good and $\Party_j$ is bad; ``(b, g)'' denotes the reverse; and ``(b, b)'' denotes the case where both parties are bad.
%Outputs within the table are written as tuples of random and adversarially chosen values, where the first value is output to $\Party_i$, and the second to $\Party_j$.
%$a, b$ are adversarially chosen values (which may or may not be distinct).
%$r, s$ are independent, uniformly random values.
%``Close'' denotes the case where $d(\pw_i, \pw_j) \leq \delta$, and ``far'' denotes the case where $d(\pw_i, \pw_j) > \delta$.
%The first two rows represent the case where the two parties are not split up by a man in the middle (so, only one indicator of closeness is necessary), and the last eight rows represent the cases where they are. 
%``Close'' indicates that the the \passwords in the relevant session were sufficiently similar; ``far'' indicates that they were not.
%In the case where the adversary split the two parties, $\bot$ indicates the adversary failed to engage in one of the two sessions.
%``(Close, close)'' then denotes the case that the man in the middle guessed sufficiently close \passwords in both sessions, and so on.

\begin{figure}
\centering

\scriptsize

%\begin{tabular}{|c||c|c|c|c|}
%\hline
%$\Frfe^{P}$ & (g, g) & (g, b) & (b, g) & (b, b) \\ \hline \hline
%close & r, r & a, b & a, b & a, b \\ \hline
%far & r, s & r, b & a, s & a, b  \\ \hline
%\end{tabular}

%\begin{tabular}{|c||c|c|c|c|}
%\hline
%$\Fsrfe^{P} = \FFAKE^{P}$ & (g, g) & (g, b) & (b, g) & (b, b) \\ \hline \hline
%close = (fresh, fresh) & r, r & a, b & a, b & a, b \\ \hline
%far = (fresh, fresh) & r, s & r, b & a, s & a, b  \\ \hline
%($\bot$, close) = (fresh, compromised) & r, b & r, b & a, b & a, b  \\ \hline
%($\bot$, far) = (fresh, interrupted) & r, s & r, b & a, s & a, b  \\ \hline
%(close, close) = (compromised, compromised) & a, b & a, b & a, b & a, b  \\ \hline
%(close, far) = (compromised, interrupted) & a, s & a, b & a, s & a, b  \\ \hline
%(close, $\bot$) = (compromised, fresh) & a, s & a, b & a, s & a, b \\ \hline
%(far, close) = (interrupted, compromised) & r, b & r, b & a, b & a, b  \\ \hline
%(far, far) = (interrupted, interrupted) & r, s & r, b & a, s & a, b  \\ \hline
%(far, $\bot$) = (interrupted, fresh) & r, s & r, b & a, s & a, b  \\ \hline
%\end{tabular}

\begin{tabular}{|C{2cm}|C{2cm}|C{2cm}||C{1.5cm}|C{1.5cm}||C{1cm}|C{1cm}|C{1cm}|}
\hline
\multicolumn{3}{|c||}{$\Fsrfe^{P}$} & \multicolumn{2}{c||}{$\FFAKE^{P}$} & \multicolumn{3}{C{3cm}|}{outputs in both $\Fsrfe^{P}$ and $\FFAKE^{P}$} \\ \hline
distance between $\firstparty$'s \password and $\secondparty$'s \password $d(\firstpw, \secondpw)$, if MITM didn't happen & distance between $\firstparty$'s \password and the adversary's $d(\firstpw, \secondpw')$, if MITM happened & distance between $\secondparty$'s \password and the adversary's $d(\firstpw', \secondpw)$, if MITM happened & $\firstparty$'s record & $\secondparty$'s record & $\firstparty$ and $\secondparty$ honest & $\firstparty$ honest, $\secondparty$ corrupt & $\firstparty$ and $\secondparty$ corrupt \\ \hline \hline
close ($\leq \delta$) & & & fresh & fresh & r, r & a, b & a, b \\ \hline
far ($> \delta$) & & & fresh & fresh & r, s & r, b & a, b \\ \hline \hline
& close ($\leq \delta$) & close ($\leq \delta$) & compromised & compromised & a, b & a, b & a, b  \\ \hline
& close ($\leq \delta$) & far ($> \delta$) & compromised & interrupted & a, s & a, b & a, b  \\ \hline
& close ($\leq \delta$) & & compromised & fresh & a, s & a, b & a, b \\ \hline
& far ($> \delta$) & close ($\leq \delta$) & interrupted & compromised & r, b & r, b & a, b  \\ \hline
& far ($> \delta$) & far ($> \delta$) & interrupted & interrupted & r, s & r, b & a, b  \\ \hline
& far ($> \delta$) & & interrupted & fresh & r, s & r, b & a, b  \\ \hline
& & close ($\leq \delta$) & fresh & compromised & r, b & r, b & a, b  \\ \hline
& & far ($> \delta$) & fresh & interrupted & r, s & r, b & a, b  \\ \hline
\end{tabular}

\caption{Output tables for %$\Frfe^{P}$ and 
$\FFAKE^{P}$ and $\Fsrfe^{P}$. $r, s$ represent random outputs, while $a, b$ represent adversarially chosen outputs.}
\label{fig:outputtables}
\end{figure}

\end{proof}

The transformation of Barak~\etal proceeds in two steps.
First, links are initialized:
\begin{enumerate}
\item
Each party generates a signing and verification key pair, and sends the verification key to its partner. 
\item 
Each party then signs the key it receives and sends the signature back.
\item
Each party verifies the signature it receives on its own verification key using the verification key it received; if the signature does not verify, it aborts.
\end{enumerate}
Second, the parties run the protocol exactly as they would over authenticated channels, signing each message with their signing key, and verifying each signature they receive.

Applying this transformation adds (1) two rounds of communication, and (2) a hash operation and a signature operation for each message, assuming the hash-and-sign paradigm is used.
