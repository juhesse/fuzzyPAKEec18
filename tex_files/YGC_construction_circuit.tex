% !TEX root = ../main.tex
% !TEX spellcheck = en-US

\subsection{An Efficient Circuit $\func$ for Hamming Distance}
\label{sec:efficientf}

The Hamming distance of two \passwords $\pw, \pw' \in\F_{\alphabetsize}^{\pwlen}$ is equal to the number of locations at which the two \passwords have the same character. 
More formally,
\[d(\pw,\pw'):=|\left\{j\,|\,\pw[j]\neq\pw'[j], j\in[\pwlen]\right\}|.\]

We design $\func$ for Hamming distance as follows:
\begin{enumerate}
\item
First, $\func$ XORs corresponding (binary) \password characters, resulting in a list of bits indicating the (in)equality of those characters.
\item
Then, $\func$ feeds those bits into a threshold gate, which returns $1$ if at least $\pwlen - \delta$ of its inputs are $0$, and returns $0$ otherwise.
$\func$ returns the output of that threshold gate, which is $1$ if and only if at least $\pwlen - \delta$ \password characters match.
\end{enumerate}

This circuit, illustrated in Figure~\ref{fig:fakecircuit}, is very efficient to garble; it only requires $\pwlen$ ciphertexts.
Below, we briefly explain this garbling.
Our explanation assumes familiarity with YGC literature~\cite[and references therein]{YGCintro}.
Briefly, garbled gadget labels~\cite{CCS:BalMalRos16} enable the evaluation of modular addition gates for free (there is no need to include any information in the garbled circuit to enable this addition).
However, for a small modulus $m$, converting the output of that addition to a binary decision requires $m-1$ ciphertexts.
We utilize garbled gadgets with a modulus of $\pwlen + 1$ in our efficient garbling as follows:

\begin{enumerate}
\item
\label{item:eqcheck}
The input wire labels encode $0$ or $1$ modulo $\pwlen + 1$.
However, instead of having those input wire labels encode the characters of the two \passwords directly, they encode the outputs of the comparisons of corresponding characters.
If the $j$th character of $\aparty$'s \password is $0$, then $\aparty$ puts the $0$ label first; 
however, if the $j$th character of $\aparty$'s \password is $1$, then $\aparty$ flips the labels. 
Then, when $\otherparty$ is using oblivious transfer to retrieve the label corresponding to her $j$th \password character, she will retrieve the $0$ label if the two characters are equal, and the $1$ label otherwise.
(Note that this pre-processing on the garbler's side eliminates the need to send $\ginp_{0,0}$ and $\ginp_{1,1}$ in Figure~\ref{fig:YGCRFE}.)
\item
\label{item:threshold}
Compute a $\pwlen$-input threshold gate, as illustrated in Figure~6 of Yakoubov~\cite{YGCintro}. %Figure~\ref{fig:ninputthreshold}.
This gate returns $0$ if the sum of the inputs is above a certain threshold (that is, if at least $\pwlen - \delta$ \password characters differ), and $1$ otherwise.
This will require $\pwlen$ ciphertexts.
\end{enumerate}

Thus, a garbling of $\func$ consists of $\pwlen$ ciphertexts.
Since \FAKE requires two such garbled circuits (Figure~\ref{fig:YGCRFE}), $2\pwlen$ ciphertexts will be exchanged.

% !TEX root = ../main.tex
% !TEX spellcheck = en-US

\newcommand{\maxx}{12}
\newcommand{\maxy}{6}
\newcommand{\inputx}{0}
\newcommand{\xorx}{\maxx/3}
\newcommand{\thresholdx}{2*\maxx/3}
\newcommand{\outputx}{\maxx}
\newcommand{\inputygap}{\maxy/7}
\newcommand{\midy}{3.5*\inputygap}

\begin{figure}[tb]
\begin{center}
\begin{tikzpicture}[
	circuit logic US,
	tiny circuit symbols,
	every circuit symbol/.style={fill=white,draw, logic gate input sep=4mm},
	scale=0.70
]

\node (c1) at (\inputx,7*\inputygap) {$\firstpw^1$};
\node (s1) at (\inputx,6*\inputygap) {$\secondpw^1$};
\node (c2) at (\inputx,5*\inputygap) {$\firstpw^2$};
\node (s2) at (\inputx,4*\inputygap) {$\secondpw^2$};
\node (c3) at (\inputx,3*\inputygap) {$\firstpw^3$};
\node (s3) at (\inputx,2*\inputygap) {$\secondpw^3$};
\node (c4) at (\inputx,\inputygap) {$\firstpw^4$};
\node (s4) at (\inputx,0) {$\secondpw^4$};

\node (key) at (\outputx,\midy) {$\{0,1\}$};

\node [xor gate] at (\xorx,6.5*\inputygap) (xor1) {eq};
\node [xor gate] at (\xorx,4.5*\inputygap) (xor2) {eq};
\node [xor gate] at (\xorx,2.5*\inputygap) (xor3) {eq};
\node [xor gate] at (\xorx,.5*\inputygap) (xor4) {eq};
\node [and gate, inputs=nnnn] at (\thresholdx,\midy) (threshold1) {threshold};

\draw (c1) -- (xor1.input 1);
\draw (s1) -- (xor1.input 2);
\draw (c2) -- (xor2.input 1);
\draw (s2) -- (xor2.input 2);
\draw (c3) -- (xor3.input 1);
\draw (s3) -- (xor3.input 2);
\draw (c4) -- (xor4.input 1);
\draw (s4) -- (xor4.input 2);

\draw (xor1.output) -- (threshold1.input 1);
\draw (xor2.output) -- (threshold1.input 2);
\draw (xor3.output) -- (threshold1.input 3);
\draw (xor4.output) -- (threshold1.input 4);

\draw (threshold1.output) -- (key);

\end{tikzpicture}
\end{center}\vspace*{-2em}
\caption{The $\func$ circuit}
\label{fig:fakecircuit}
\end{figure}

\paragraph{Larger \Password Characters.}

If larger \password characters are used, then Step~\ref{item:eqcheck} above needs to change to check (in)equality of the larger characters instead of bits.
Step~\ref{item:threshold} will remain the same.
There are several ways to perform an (in)equality check on characters in $\F_{\alphabetsize}$ for $\alphabetsize \geq 2$:
\begin{enumerate}
\item 
Represent each character in terms of bits.
Step~\ref{item:eqcheck} will then consist of XORing corresponding bits, and taking an OR or the resulting XORs of each character to get negated equality.
This will take an additional $\pwlen \log(\alphabetsize)$ ciphertexts for every \password character.
\item
Use garbled gadget labels from the outset. 
We will require a larger OT ($1$-out-of-$\alphabetsize$ instead of $1$-out-of-$2$), but nothing else will change.
\end{enumerate}
