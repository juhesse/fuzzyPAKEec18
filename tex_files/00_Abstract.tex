% !TeX root = ../main.tex
% !TEX root = ../main.tex
% !TEX spellcheck = en-US
Consider key agreement by two parties who start out knowing a common secret (which we refer to as ``\password'', a generalization of ``password''),
but face two complications: 
(1) the \password may come from a low-entropy distribution, and 
(2) the two parties' copies of the \password may have some noise, and thus not match exactly.
We provide the first efficient and general solutions to this problem that enable, for example, key agreement based on commonly used biometrics such as iris scans.\\
\\
The problem of key agreement with each of these complications individually has been well studied in literature.
Key agreement from low-entropy shared \passwords is achieved by \emph{password-authenticated key exchange} (\PAKE), and 
key agreement from noisy but high-entropy shared \passwords is achieved by information-reconciliation protocols as long as the two secrets are ``close enough.''
However, the problem of key agreement from noisy low-entropy \passwords has never been studied.\\
\\
We introduce (universally composable) \emph{fuzzy password-authenticated key exchange} (\FAKE), which solves exactly this problem.
\FAKE does not have any entropy requirements for the \passwords, and enables secure key agreement as long as the two \passwords are ``close'' for some notion of closeness.
We also give two constructions. 
The first construction achieves our \FAKE definition for any (efficiently computable) notion of closeness, including those that could not be handled before even in the high-entropy setting.
It uses Yao's garbled circuits in a way that is only two times more costly than their use against semi-honest adversaries, but that guarantees security against malicious adversaries.
The second construction is more efficient, but achieves our \FAKE definition only for \passwords with low Hamming distance.
It builds on very simple primitives: robust secret sharing and \PAKE.

\begin{mdframed}[tikzsetting={draw=red,ultra thick},skipabove=5pt]
\begin{minipage}{\textwidth}
 \textbf{Disclaimer:} Theorem \ref{theorem:fake2} in Section \ref{sec:secrssfpake} is not correct (thanks to Johannes Ernst from University St. Gallen, CH, for pointing this out to us): a man-in-the-middle attacker can test individual password bits, and the RSS-based fuzzy PAKE construction hence is not secure. We will soon update this document with information on how to fix the protocol to prevent these attacks. 
\end{minipage}
\end{mdframed}
