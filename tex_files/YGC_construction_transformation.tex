% !TEX root = ../main.tex
% !TEX spellcheck = en-US

\subsubsection{From Split Randomized Fuzzy Equality to \FAKE}
\label{sec:ygcfaketransformation}

The Randomized Fuzzy Equality (RFE) functionality $\Frfe^{P}$ assumes authenticated channels, which an $\FAKE$ protocol cannot do.
In order to adapt RFE to our setting, we use the split functionality transformation defined by Barak~\etal~\cite{C:BCLPR05}.
Barak~\etal provide a generic transformation from protocols which require authenticated channels to protocols which do not.
In the ``transformed'' protocol, an adversary can engage in two separate instances of the protocol with the sender and receiver, and they will not realize that they are not talking to one another.
However, it does guarantee that the adversary cannot do anything beyond this attack.
In other words, it provides ``session authentication'', meaning that each party is guaranteed to carry out the entire protocol with the same partner, but not ``entity authentication'', meaning that the identity of the partner is not guaranteed.

Barak~\etal achieve this transformation in three steps.
First, the parties generate signing and verification keys, and send one another their verification keys.
Next, the parties sign the list of all keys they have received (which, in a two-party protocol, consists of only one key), sign that list, and send both list and signature to all other parties.
Finally, they verify all of the signatures they have received.
After this process --- called ``link initialization'' --- has been completed, the parties use those public keys they have exchanged to authenticate subsequent communication. 

We describe the Randomized Fuzzy Equality Split Functionality in Figure~\ref{fig:func-SRFE}.
It is simplified from Figure 1 in Barak~\etal~\cite{C:BCLPR05} because we only need to consider two parties and static corruptions.

\begin{figure}[tbp]
  \centering
  \begin{fboxenv}
    \begin{minipage}{0.95\textwidth}
      The functionality $\Fsrfe^{P}$ is parameterized by a security parameter~$\secparam$.
      It interacts with an adversary~$\Sx$ and two parties $\firstparty$ and $\secondparty$ via the following queries:\\[-1.8em]
      \begin{itemize}
      \item \textbf{Initialization}
        \begin{itemize}
        \item
        \textbf{Upon receiving a query $(\Init, \sid)$ from a party $\aparty \in \{\firstparty,\secondparty\}$}, 
        send $(\Init, \sid, \aparty)$ to the adversary $\Sx$.
        \item
        \textbf{Upon receiving a query $(\Init, \sid, \aparty, H, \sid_H)$ from the adversary $\Sx$}:
          \begin{itemize}
          \item
          Verify that $H \subseteq \{\firstparty, \secondparty\}$, that $\aparty \in H$, and that if a previous set $H'$ was recorded, either (1) $H \cap H'$ contains only corrupted parties and $\sid_H \neq \sid_{H'}$, or (2) $H = H'$ and $\sid_H = \sid_{H'}$.
          \item
          If verification fails, do nothing.
          \item 
          Otherwise, 
          record the pair $(H, \sid_H)$ (if it was not already recorded), 
          output $(\Init, \sid, \sid_H)$ to $\aparty$, and
          locally initialize a new instance of the original RFE functionality $\Frfe$ denoted $H\Frfe^{P}$, letting the adversary play the role of $\{\firstparty, \secondparty\} - H$ in $H\Frfe^{P}$.
          \end{itemize}
        \end{itemize}
      \item \textbf{RFE}
        \begin{itemize}
        \item
        \textbf{Upon receiving a query from a party $\aparty \in \{\firstparty, \secondparty\}$,}
        find the set $H$ such that $\aparty \in H$, and forward the query to $H\Frfe^{P}$.
        Otherwise, ignore the query.
        \item
        \textbf{Upon receiving a query from the adversary $\Sx$ on behalf of $\aparty$ corresponding to set $H$,}
        if $H\Frfe^{P}$ is initialized and $\aparty \not\in H$, then forward the query to $H\Frfe^{P}$.
        Otherwise, ignore the query.
        \end{itemize}
      \end{itemize}
    \end{minipage}
  \end{fboxenv}
  \caption{Functionality $\Fsrfe^{P}$}
  \label{fig:func-SRFE}
\end{figure}

It turns out that $\Fsrfe^{P}$ is enough to realize $\FFAKE^{P}$.
In fact, the protocol $\Prfe$ with the split functionality transformation directly realizes $\FFAKE^{P}$.
In \appref{sec:ygcfakeproof}, we prove that this is the case.
