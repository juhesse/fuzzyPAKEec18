% !TEX root = ../main.tex
% !TEX spellcheck = en-US
\sophia[inline]{How can we use the OT one way, encryption the other way trick from Canetti~\etal~\cite{PKC:CDVW12}?}
In this section, we describe a protocol realizing $\FAKE^{P}$ that uses Yao's garbled circuits~\cite{FOCS:Yao86}.
We briefly introduce this primitive in Sec.~\ref{sec:YGCbackground} and refer to Yakoubov~\cite{YGCintro} for a more thorough introduction.

The Yao's garbled circuit-based $\FAKE$ construction has two advantages:
\begin{enumerate}
\item 
It is more flexible than other approaches; any notion of distance that can be efficiently computed by a circuit can be used.
In Section~\ref{sec:efficientf}, we describe a suitable circuit for Hamming distance.
The total size of this circuit is $O(\pwlen)$, where $\pwlen$ is the length of the \passwords used.
Edit distance is slightly less efficient, and uses a circuit whose total size is $O(\pwlen^2)$.
\item 
There is no gap between the distances required for functionality and security --- that is, there is no leakage about the \passwords used unless they are similar enough to agree on a key. 
In other words, $\delta = \gamma$.
\end{enumerate}

Informally, the construction involves the garbled evaluation of a circuit that takes in two \passwords as input, and computes whether their distance is less than $\delta$.
Because Yao's garbled circuits are only secure against semi-honest garblers, we cannot simply have one party do the garbling and the other party do the evaluation.
A malicious garbler could provide a garbling of the wrong function --- maybe even a constant function --- which would result in successful key agreement even if the two \passwords are very different.
However, as suggested by Mohassel~\etal~\cite{PKC:MohFra06b} and Huang~\etal~\cite{SP:HuaKatEva12}, since a malicious evaluator (unlike a malicious garbler) cannot compromise the computation, by performing the protocol twice with each party playing each role once, we can protect against malicious behavior.
They call this the \emph{dual execution} protocol.

The dual execution protocol has the downside of allowing the adversary to specify and receive a single additional bit of leakage. 
It is important to note that because of this, dual execution cannot directly be used to instantiate \fPAKE, because a single bit of leakage can be too much when the entropy of the \passwords is low to begin with --- a few adversarial attempts will uncover the entire \password.
Our construction is as efficient as that of Mohassel~\etal and Huang~\etal, while guaranteeing no leakage to a malicious adversary in the case that the \passwords used are not close.
We describe how we achieve this in Section~\ref{sec:YGCbackgroundOurs}. 

%Generalizing this beyond fuzzy PAKE, by adding a secure comparison at the end of our protocol in the style of Mohassel~\etal and Huang~\etal, our technique can be used to evaluate any boolean function in such a way that security holds as long as the correct output is `no'. 
%(That is, if the correct output is `no' then an adversary cannot convince an honest party that it is `yes', and cannot learn any additional bits. 
%If the correct output is `yes', then an adversary can convince an honest party that it is `no', and can also learn an additional bit.) 
%This is sufficient in any scenario where finding a `yes' instance is enough to establish mutual trust.

Due to the symmetric layout of our construction, we skip all $\role$ tags in this section.




% !TEX root = ../main.tex
% !TEX spellcheck = en-US

\subsection{Building Blocks}
\label{sec:YGCbackground}

In Section~\ref{sec:YGCbackgroundOT}, we briefly review oblivious transfer.
In Section~\ref{sec:YGCbackgroundYGC}, we review Yao's Garbled Circuits.
In Section~\ref{sec:YGCbackgroundOurs}, we describe in more detail our take on the dual execution protocol, and how we avoid leakage to the adversary when the \passwords used are dissimilar.

\subsubsection{Oblivious Transfer (OT)}
\label{sec:YGCbackgroundOT}

Informally, $1$-out-of-$2$ Oblivious Transfer (see Chou and Orlandi~\cite{LC:ChoOrl15} and citations therein) enables one party (the sender) to transfer exactly one of two secrets to another party (the receiver). 
The receiver chooses (by index 0 or 1) which secret she wants. 
The security of the OT protocol guarantees that the sender does not learn this choice bit, and 
the receiver does not learn anything about the other secret.

\subsubsection{Yao's Garbled Circuits (YGC)}
\label{sec:YGCbackgroundYGC}

In this section, we give a brief introduction to Yao's garbled circuits~\cite{FOCS:Yao86}.
We refer to Yakoubov~\cite{YGCintro} for a more detailed description, as well as a summary of some of the Yao's garbled circuits optimizations~\cite{STOC:BeaMicRog90,ICALP:KolSch08,AC:PSSW09,C:KolMohRos14,EC:ZahRosEva15,CCS:BalMalRos16}.
Informally, Yao's garbled circuits are an asymmetric secure two-party computation scheme.
They enable two parties with sensitive inputs (in our case, \passwords) to compute a joint function of their inputs (in our case, an augmented version of similarity) without revealing any additional information about their inputs.
One party ``garbles'' the function they wish to evaluate, and the other evaluates it in its garbled form.

Below, we summarize the garbling scheme formalization of Bellare~\etal~\cite{CCS:BelHoaRog12}, which is a generalization of YGC.

\paragraph{Functionality.}

A garbling scheme $\gbscheme$ consists of a tuple of four polynomial-time algorithms $(\gb, \en, \ev, \de)$:
\begin{enumerate}
\item $\gb(1^{\secparam}, \func) \to (\gfunc, \eninp, \deinp)$.
The garbling algorithm $\gb$ 
takes in the security parameter $\secparam$ and a circuit $\func$, 
and returns a garbled circuit $\gfunc$, encoding information $\eninp$,
and decoding information $\deinp$.
\item $\en(\eninp, \inp) \to \ginp$.
The encoding algorithm $\en$ takes in the encoding information $\eninp$ and an input $\inp$, and returns a garbled input $\ginp$.
\item $\ev(\gfunc, \ginp) \to \goutp$.
The evaluation algorithm $\ev$ takes in the garbled circuit $\gfunc$ and the garbled input $\ginp$, and returns a garbled output $\goutp$.
\item $\de(\deinp, \goutp) \to \outp$.
The decoding algorithm $\de$ takes in the decoding information $\deinp$ and the garbled output $\goutp$, and returns the plaintext output $\outp$.
\end{enumerate}
A garbling scheme $\gbscheme = (\gb, \en, \ev, \de)$ is \emph{projective} if the encoding information $\eninp$ consists of $2 \numinputs$ \emph{wire labels} (each of which is essentially a random string), where $\numinputs$ is the number of input bits. 
Two wire labels are associated with each bit of the input; one wire label corresponds to the event of that bit being $0$, and the other corresponds to the event of that bit being $1$.
The garbled input includes only the wire labels corresponding to the actual values of the input bits.
In projective schemes, in order to give the evaluator the garbled input she needs for evaluation,
the garbler can send her all of the wire labels corresponding to the garbler's input.
The evaluator can then use OT to retrieve the wire labels corresponding to her own input. 

Similarly, we call a garbling scheme \emph{output-projective} if decoding information $\deinp$ consists of two labels for each output bit, one corresponding to each possible value of that bit.
The garbling schemes used in this paper are both projective and output-projective.

\paragraph{Correctness.}
Informally, a garbling scheme $(\gb, \en, \ev, \de)$ is \emph{correct} if it always holds that $\de(\deinp, \ev(\gfunc, \en(\eninp, \inp))) = \func(\inp)$.

\paragraph{Security.}
\label{sec:ygc_security}
Bellare \etal~\cite{CCS:BelHoaRog12} describe three security notions for garbling schemes: 
\emph{obliviousness}, \emph{privacy} and \emph{authenticity}.
Informally, a garbling scheme $\gbscheme = (\gb, \en, \ev, \de)$ is \emph{oblivious} if a garbled function $\gfunc$ and a garbled input $\ginp$ do not reveal anything about the input $\inp$. 
It is \emph{private} if additionally knowing the decoding information $\deinp$ reveals the output $\outp$, but does not reveal anything more about the input $\inp$.
It is \emph{authentic} if an adversary, given $\gfunc$ and $\ginp$, cannot find a garbled output $\goutp' \neq \ev(\gfunc, \ginp)$ which decodes without error.

In \appref{sec:garbledoutputrandomness}, we define a new property of output-projective garbling schemes called \emph{garbled output randomness}.
Informally, it states that even given one of the output labels, the other should be indistinguishable from random.

\subsubsection{Malicious Security: A New Take on Dual Execution with Privacy-Correctness Tradeoffs}
\label{sec:YGCbackgroundOurs}

While Yao's garbled circuits are naturally secure against a malicious evaluator, they have the drawback of being insecure against a malicious garbler.
A garbler can ``mis-garble'' the function, either replacing it with a different function entirely or causing an error to occur in an informative way (this is known as ``selective failure'').

Typically, malicious security is introduced to Yao's garbled circuits by using the cut-and-choose transformation~\cite{JC:LinPin15,C:Lindell13,C:HuaKatEva13}. 
To achieve a $2^{-\secparam}$ probability of cheating without detection, the parties need to exchange $\secparam$ garbled circuits~\cite{C:Lindell13}.\footnote{
There are techniques~\cite{C:LinRiv14} that improve this number in the amortized case when many computations are done --- however, this does not fit our setting.}
Some of the garbled circuits are ``checked'', and the rest of them are evaluated, their outputs checked against one another for consistency. 
Because of the factor of $\secparam$ computational overhead, though, cut-and-choose is expensive, and too heavy a tool for $\FAKE$.
Other, more efficient transformations such as LEGO~\cite{TCC:NieOrl09} and authenticated garbling~\cite{CCS:WanRanKat17a} exist as well, but those rely heavily on pre-processing, which cannot be used in $\FAKE$ since it requires advance interaction between the parties.

Mohassel~\etal~\cite{PKC:MohFra06b} and Huang~\etal~\cite{SP:HuaKatEva12} suggest an efficient transformation known as ``dual execution'':
each party plays each role (garbler and evaluator) once, and then the two perform a comparison step on their outputs in a secure fashion. 
Dual execution incurs only a factor of $2$ overhead over semi-honest garbled circuits. 
However, it does not achieve fully malicious security. 
It guarantees correctness, but reduces the privacy guarantee by allowing
a malicious garbler to learn one bit of information of her choice. 
Specifically, if a malicious garbler garbles a wrong circuit, she can use the comparison step to learn one bit about the output of this wrong circuit on the other party's input.
This one extra bit of information could be crucially important, violating the privacy of the evaluator's input in a significant way. %especially if the input is used in more than one computation.

We introduce a tradeoff between correctness and privacy for boolean functions. 
For one of the two possible outputs (without loss of generality, `0'), we restore full privacy at the cost of correctness.
The new privacy guarantee is that if the correct output is `0', then  a malicious adversary cannot learn anything beyond this output, but if the correct output is `1', then she can learn a single bit of her choice. 
The new correctness guarantee is that a malicious adversary can cause the computation  that should output `1' to output `0' instead, but not the other way around.
Our privacy--correctness tradeoff is summarized in Figure~\ref{fig:tradeoff}.

\begin{figure}
\centering
\begin{tabular}{|cc|c|c|c|}
\hline
\multirow{3}{*}{\rotatebox{90}{\centering \cite{PKC:MohFra06b}}} & \multirow{3}{*}{\rotatebox{90}{\centering \cite{SP:HuaKatEva12}}} & Correct Output & Computed Output & Privacy \\ \hhline{~~---} \hhline{~~---}
& & 1  & 1 OR `cheating' & 1-bit leakage \\ \hhline{~~---}
& & 0 & 0 OR `cheating' & 1-bit leakage \\ \hline \hline \hline
\multirow{3}{*}{\rotatebox{90}{\centering Our}} & \multirow{3}{*}{\rotatebox{90}{\centering Protocol}} & Correct Output & Computed Output & Privacy \\ \hhline{~~---} \hhline{~~---}
& & 1 & 1 OR 0 & 1-bit leakage \\ \hhline{~~---}
& & 0 & 0 & full privacy \\ \hline
\end{tabular}
\caption{The Privacy-Correctness Tradeoff of Dual Execution Protocols for Boolean Functions}
\label{fig:tradeoff}
\end{figure}

The main idea of dual execution is to have the two parties independently evaluate one another's circuits, learn the output values, and compare the output labels using a secure comparison protocol.
This comparison step is simply a check for malicious behavior;
if comparison fails, then honest party $\aparty$ learns that $\otherparty$ cheated. 
If the comparison step succeeded, $\otherparty$ might still have cheated --- and gleaned an extra bit of information --- but $\aparty$ is assured that she has the correct output.

In our construction, however, the parties need not learn the output values before the comparison.
Instead, the parties can compare output labels \emph{assuming} an output of `1', and if the comparison fails, the output is determined to be `0'. 
More formally, 
let $\deinp_{\firstindex}[0]$, $\deinp_{\firstindex}[1]$ be the two output labels corresponding to $\firstparty$'s garbled circuit, 
and $\deinp_{\secondindex}[0]$, $\deinp_{\secondindex}[1]$ be the two output labels corresponding to $\secondparty$'s circuit.
Let $\goutp_{\secondindex} \in [\deinp_{\secondindex}[0], \deinp_{\secondindex}[1]]$ be the output label learned by $\firstparty$ as a result of evaluation, 
and $\goutp_{\firstindex} \in [\deinp_{\firstindex}[0], \deinp_{\firstindex}[1]]$ be the label learned by $\secondparty$.
The two parties securely compare $(\deinp_{\firstindex}[1], \goutp_{\secondindex})$ to $(\goutp_{\firstindex}, \deinp_{\secondindex}[1])$;
if the comparison succeeds, the output is ``1''.

Whereas in dual execution the comparison step is just a sanity check, here it determines the actual computation output.
If the correct output is `1', a cheating $\otherparty$ can still learn one bit of information by mis-garbling her circuit; depending on the output of the mis-garbled circuit, the comparison step will either succeed or fail.
If the comparison fails, $\aparty$ will accept an incorrect output of `0', and never be aware that $\otherparty$ cheated.
If the correct output is `0', however, there is nothing $\otherparty$ can do to cause the comparison step to succeed, since in order to do this, she would need to use the second output label $\deinp_{\aindex}[1]$ as an input.
Since the true output was `0', and thus $\goutp_{\aindex} = \deinp_{\aindex}[0]$,
by the garbled output randomness property of the garbling scheme, $\otherparty$ can't even distinguish $\deinp_{\aindex}[1]$ from random.

Our privacy--correctness tradeoff is perfect for $\FAKE$.
If the parties' inputs are similar, learning a bit of information about each other's inputs is not problematic, since arguably the small amount of noise in the inputs is a bug, not a feature.
If the parties' inputs are not similar, however, we are guaranteed to have no leakage at all. 
We pay for the lack of leakage by allowing a malicious party to force an authentication failure even when authentication should succeed, which either party can do anyway simply by providing an incorrect input.

In Section~\ref{sec:rfeprot}, we describe our Yao's garbled circuit-based $\FAKE$ protocol in detail.
Note that in this protocol, we omit the final comparison step; instead, we use the output lables ($(\deinp_{\firstindex}[1], \goutp_{\secondindex})$ and $(\goutp_{\firstindex}, \deinp_{\secondindex}[1])$) to compute the agreed-upon key directly (via XOR).


\subsection{Construction}
\label{sec:YGCconstruction}

Building a $\FAKE$ from YGC and OT is not straightforward, since all constructions of OT assume authenticated channels, and \FAKE (or \PAKE) is designed with unauthenticated channels in mind.
We therefore follow the framework of Canetti~\etal~\cite{PKC:CDVW12}, who build a UC secure \PAKE protocol using OT.
We first build our protocol assuming authenticated channels, and then apply the generic transformation of Barak~\etal~\cite{C:BCLPR05} to adapt it to the unauthenticated channel setting.
More formally, we proceed in three steps:
\begin{enumerate}
\item 
First, in Section~\ref{sec:rfe}, we define a randomized fuzzy equality-testing functionality $\Frfe$, which is analogous to the randomized equality-testing functionality of Canetti~\etal 
\item
In Section~\ref{sec:rfeprot}, we build a protocol that securely realizes $\Frfe$ in the OT-hybrid model, assuming authenticated channels.
\item
In Section~\ref{sec:ygcfaketransformation}, we apply the transformation of Barak~\etal to our protocol.
This results in a protocol that realizes the ``split'' version of functionality $\Frfe^P$, which we show to be enough to implement to $\FAKE^{P}$.
Split functionalities, which were introduced by Barak~\etal, adapt functionalities which assume authenticated channels to an unauthenticated channels setting. 
The only additional ability an adversary has in a split functionality is the ability to execute the protocol separately with the participating parties.
\end{enumerate}


% !TEX root = ../main.tex
% !TEX spellcheck = en-US

\subsubsection{The Randomized Fuzzy Equality Functionality}
\label{sec:rfe}

Figure~\ref{fig:func-RFE} shows the randomized fuzzy equality functionality $\Frfe^{P}$, which is essentially what $\FFAKE^{P}$ would look like assuming authenticated channels.
The primary difference between $\Frfe^{P}$ and $\FFAKE^{P}$ is that the only \password guesses allowed by $\Frfe^{P}$ are the ones actually used as protocol inputs; this limits the adversary to guessing by corrupting one of the participating parties, not through man in the middle attacks.
Like in $\FFAKE^{P}$, if a \password guess is ``similar enough'', the entire \password is leaked. This leakage could be replaced with any other leakage from Section~\ref{sec:model}; $\Frfe$ would leak the correctness of the guess, $\Frfe^{M}$ would leak which characters are the same between the two \passwords, etc.

Note that, unlike the randomized equality functionality in the work of Canetti~\etal~\cite{PKC:CDVW12}, $\FFAKE^P$ has a $\TestPwd$ interface. 
This is because $\NewKey$ does not return the necessary leakage to an honest user.
So, an interface enabling the adversary to retrieve additional information is necessary.

\begin{figure}[tb]
  \centering
  \begin{fboxenv}
    \begin{minipage}{0.95\textwidth}
      The functionality $\Frfe$ is parameterized by a security parameter~$\secparam$ and a tolerance $\delta$.
      It interacts with an adversary~$\Sx$ and two parties $\firstparty$ and $\secondparty$ via the following queries:\\[-1.8em]
      \begin{itemize}
      \item
        \textbf{Upon receiving a query
        \mathversion{bold}$(\NewSession,\sid,\apw)$ from party $\aparty \in \{\firstparty,\secondparty\}$\mathversion{normal}:}
        \begin{itemize}
          \item Send $(\NewSession,\sid,\aparty)$ to~$\Sx$;
          \item If this is the first \NewSession query,
          or if this is the second \NewSession query and there is a record~$(\otherparty,\otherpw)$,
          then record~$(\aparty,\apw)$.
        \end{itemize}
      \item 
        \textbf{Upon receiving a query
          \mathversion{bold}$(\TestPwd,\sid,\aparty)$ from the adversary~$\Sx$, $\aparty \in \{\firstparty,\secondparty\}$\mathversion{normal}:} \\
        If records of the form $(\firstparty, \firstpw)$ and $(\secondparty, \secondpw)$ do not exist, if $\otherparty$ is not corrupted, or this is not the first $\TestPwd$ query for $\aparty$, ignore this query.
        Otherwise, if $d(\firstpw, \secondpw) \leq \delta$, send $\apw$ to the adversary $\Sx$.
      \item
        \textbf{Upon receiving a query
          \mathversion{bold}$(\NewKey,\sid,\aparty,\sk)$ from the adversary~$\Sx$, $\aparty \in \{\firstparty,\secondparty\}$\mathversion{normal}:} \\
        If there are no records of the form~$(\aparty,\apw)$ and $(\otherparty, \otherpw)$, or if this is not the first \NewKey query for~$\aparty$, then ignore this query. 
        Otherwise:
        \begin{itemize}
        \item If at least one of the following is true, then output~$(\sid,\sk)$ to party~$\aparty$.
          \begin{itemize}
           \item $\aparty$ is corrupted
           \item $\otherparty$ is corrupted and $d(\firstpw,\secondpw) \leq \delta$
          \end{itemize}
        \item If both parties are honest, $d(\firstpw,\secondpw) \leq \delta$, and a key~$\otherkey$ was sent to~$\otherparty$,
          then output~$(\sid,\otherkey)$ to~$\Party_i$.
        \item In any other case, pick a new random key~$\akey$ of length~$\secparam$ and send~$(\sid,\akey)$ to~$\aparty$.
        \end{itemize}
      \end{itemize}
    \end{minipage}
  \end{fboxenv}
  \caption{Ideal Functionality $\Frfe^{P}$ for Randomized Fuzzy Equality}
  \label{fig:func-RFE}
\end{figure}


% !TEX root = ../main.tex
% !TEX spellcheck = en-US

\subsubsection{A Randomized Fuzzy Equality Protocol}
\label{sec:rfeprot}

\begin{figure}[tb]
  \centering
  \scriptsize
   \begin{fboxenv}
     \begin{tabular}{lrcl}
     & $\firstparty(\firstpw \in \{0,1\}^{\pwlen})$ &   & $\secondparty(\secondpw \in \{0,1\}^{\pwlen})$ \\ \hline \\
    1 & $(\gfunc_{\firstindex}, \eninp_{\firstindex}, \deinp_{\firstindex}) \gets \gb(1^{\secparam}, \func)$ & & $(\gfunc_{\secondindex}, \eninp_{\secondindex}, \deinp_{\secondindex}) \gets \gb(1^{\secparam}, \func)$\\
    & parse $\eninp_{\firstindex} = (\eninp_{\firstindex, \firstindex}, \eninp_{\firstindex, \secondindex})$ & & parse $\eninp_{\secondindex} = (\eninp_{\secondindex, \secondindex}, \eninp_{\secondindex, \firstindex})$ \\ \\
    2 & & perform two OTs in parallel: & \\ \\
    & (sender) & $\xrightarrow{\makebox[1cm]{$\eninp_{\firstindex, \secondindex}$}} \hspace{1cm} \xleftarrow{\makebox[1cm]{$\secondpw$}}$ & (receiver) \\ 
    & & $\fbox{OT}$ & \\
    & & $\hspace{2.2cm} \xrightarrow[{\makebox[1cm]{$\ginp_{\firstindex, \secondindex}=\en(\eninp_{\firstindex,\secondindex},\secondpw)$}}]{}$ &  \\ 
    &&& \\
    & (receiver) & $\xrightarrow{\makebox[1cm]{$\firstpw$}} \hspace{1cm} \xleftarrow{\makebox[1cm]{$\eninp_{\secondindex, \firstindex}$}}$ & (sender) \\ 
    & & $\fbox{OT}$ & \\
    & & $\xleftarrow[{\makebox[1cm]{$\ginp_{\secondindex, \firstindex}=\en(\eninp_{\secondindex,\firstindex},\firstpw)$}}]{} \hspace{2.2cm}$ &  \\ \\
    %2 & $\ginp_{\secondindex,\firstindex} = \en(\eninp_{\secondindex,\firstindex}, \firstpw)$ &  $\LRsflow{}{\mbox{OT}}$ & $\ginp_{\firstindex,\secondindex} = \en(\eninp_{\firstindex,\secondindex}, \secondpw)$ \\
    3 & $\ginp_{\firstindex,\firstindex} = \en(\eninp_{\firstindex,\firstindex}, \firstpw)$ & & $\ginp_{\secondindex,\secondindex} = \en(\eninp_{\secondindex,\secondindex}, \secondpw)$ \\
    4 & &  $\LRsbflow{\ginp_{\firstindex,\firstindex}, \gfunc_\firstindex}{\ginp_{\secondindex,\secondindex}, \gfunc_{\secondindex}}$ & \\
     & $\ginp_{\secondindex} = (\ginp_{\secondindex,\secondindex}, \ginp_{\secondindex,\firstindex})$ &  & $\ginp_{\firstindex} = (\ginp_{\firstindex,\firstindex}, \ginp_{\firstindex,\secondindex})$ \\
    5 & $\goutp_{\secondindex} = \ev(\gfunc_{\secondindex}, \ginp_{\secondindex})$ & & $\goutp_{\firstindex} = \ev(\gfunc_{\firstindex}, \ginp_{\firstindex})$ \\
    6 & $\key_{\firstindex, wrong} = \deinp_{\firstindex}[0]$ & & $\key_{\secondindex, wrong} = \deinp_{\secondindex}[0]$ \\
    7 & $\key_{\firstindex, correct} = \deinp_{\firstindex}[1]$ & & $\key_{\secondindex, correct} = \deinp_{\secondindex}[1]$ \\
    8 & $\firstkey = \key_{\firstindex, correct} \oplus \goutp_{\secondindex}$ & & $\secondkey = \key_{\secondindex, correct} \oplus 
   \goutp_{\firstindex}$ \\
    \end{tabular}
   \end{fboxenv}
  \caption{A Protocol $\Prfe$ Realizing $\Frfe^{P}$ using Yao's garbled circuits and an Ideal OT Functionality.
  If at any point an expected message fails to arrive (or arrives malformed), the parties output a random key.
  Subscripts are used to indicate who produced the object in question.
  If a double subscript is present, the second subscript indicates whose data the object is meant for use with. 
  For instance, a double subscript $0, 1$ denotes that the object was produced by party $\firstparty$ for use with $\secondparty$'s data; $\eninp_{\firstindex, \secondindex}$ is encoding information produced by $\firstparty$ to encode $\secondparty$'s \password.
Note that we abuse notation by encoding inputs to a single circuit separately; the input to $\firstparty$'s circuit corresponding to $\firstpw$ is encoded by $\firstparty$ locally, and the input corresponding to $\secondpw$ is encoded via OT. 
 For any projective garbling scheme, this is not a problem.}
  \label{fig:YGCRFE}
\end{figure}

\expl{If we use Yao's garbled circuits in a black-box way, we need a random oracle query at the end instead of an XOR, because the other key isn't guaranteed to be indistinguishable from random to the adversary; it's just guaranteed to be hard to guess with non-negligible probability.}

In Figure~\ref{fig:YGCRFE} we introduce a protocol $\Prfe$ that securely realizes $\Frfe^{P}$ using Yao's garbled circuits.
Garbled circuits are secure against a malicious evaluator, but only a semi-honest garbler; however, we obtain security against malicious adversaries by having each party play each role once, as describe in Section~\ref{sec:YGCbackgroundOurs}.
In more detail, both parties $\aparty \in \{\firstparty, \secondparty\}$ proceed as follows:

\begin{enumerate}
\item 
$\aparty$ garbles the circuit $\func$ that takes in two \passwords $\firstpw$ and $\secondpw$, and returns `1' if $d(\firstpw, \secondpw) \leq \delta$ and `0' otherwise.
Section~\ref{sec:efficientf} describes how $\func$ can be designed efficiently for Hamming distance. 
Instead of using the output of $\func$ (`0' or `1'), we will use the garbled output, also referred to as an \emph{output label} in an output-projective garbling scheme.
The possible output labels are two random strings --- one corresponding to a `1' output (we call this label $\key_{\aindex, correct} = \deinp_{\aindex}[1]$), and one corresponding to a `0' output (we call this label $\key_{\aindex, wrong} = \deinp_{\aindex}[0]$).
\item
$\aparty$ uses OT to retrieve the input labels from $\otherparty$'s garbling that correspond to $\aparty$'s \password.
(Similarly, $\aparty$ uses OT to send $\otherparty$ the input labels from her own garbling that correspond to $\otherparty$'s \password.)
\item 
$\aparty$ sends $\otherparty$ her garbled circuit, together with the input labels from her garbling that correspond to her own \password.
After this step, $\aparty$ should have $\otherparty$'s garbled circuit and a garbled input consisting of input labels corresponding to the bits of the two \passwords.
\item 
$\aparty$ evaluates $\otherparty$'s garbled circuit, and obtains an output label $\goutp_{\otherindex}$ (where $\goutp_{\otherindex} \in \{\key_{\otherindex, correct}, \key_{\otherindex, wrong}\}$) .
\item 
$\aparty$ outputs $\key_{\aindex} = \key_{\aindex, correct} \xor \goutp_{\otherindex}$.
\end{enumerate}

The natural question to ask is why $\Prfe$ only realizes $\Frfe^{P}$, and not a stronger functionality with less leakage.
We argue this assuming (without loss of generality) that $\secondparty$ is corrupted.
$\Prfe$ cannot realize a functionality that leaks less than the full \password $\firstpw$ to $\secondparty$ if $d(\firstpw, \secondpw) \leq \delta$;
intuitively, this is because if $\secondparty$ knows a \password $\secondpw$ such that $d(\firstpw, \secondpw) \leq \delta$,
$\secondparty$ can extract the actual \password $\firstpw$, as follows.
% \julia[inline]{this is actually a bit misleading since we only care about what is leaked in \emph{one} execution of the protocol. recovering the whole \password here takes several protocol executions. what is important is that $\secondparty$ can extract an *arbitrary* bit, and this can only be simulated if \Sim knows the whole \password.}
% \sophia[inline]{I make this argument as well in the next paragraph. However, I think that this (slightly misleading) argument is important for intuition... simulatability is the technical reason, but this is a real-world reason.}
% \julia[inline]{Ok, I'm fine leaving it like it is.}
If $\secondparty$ plays the role of OT receiver and garbled circuit evaluator honestly, 
$\firstparty$ and $\secondparty$ will agree on $\key_{\firstindex, correct}$.
$\secondparty$ can then mis-garble a circuit that returns $\key_{\secondindex, correct}$ if the first bit of $\firstpw$ is $0$, and $\key_{\secondindex, wrong}$ if the first bit of $\firstpw$ is $1$. 
By testing whether the resulting keys $\firstkey$ and $\secondkey$ match (which $\secondparty$ can do in subsequent protocols where the key is used), $\secondparty$ will be able to determine the actual first bit of $\firstpw$.
$\secondparty$ can then repeat this for the second bit, and so on, extracting the entire \password $\firstpw$.
Of course, if $\secondparty$ does \emph{not} know a sufficiently close $\secondpw$, $\secondparty$ will not be able to perform these tests, because the keys will not match no matter what circuit $\secondparty$ garbles.

More formally, if $\secondparty$ knows a \password $\secondpw$ such that $d(\firstpw, \secondpw) \leq \delta$ and carries out the mis-garbling attack described above, then in the real world, the keys produced by $\firstparty$ and $\secondparty$ either will or will not match based on some predicate $p$ of $\secondparty$'s choosing on the two \passwords $\firstpw$ and $\secondpw$.
Therefore, in the ideal world, the keys should also match or not match based on $p(\firstpw, \secondpw)$; otherwise, the environment will be able to distinguish between the two worlds. 
In order to make that happen, since the simulator does not know the predicate $p$ in question, the simulator must be able to recover the entire \password $\firstpw$ (given a sufficiently close $\secondpw$) through the $\TestPwd$ interface.

\begin{theorem}
\label{theorem:YGCRFE}
If $(\gb, \en, \ev, \de)$ is %the (projective and output-projective) secure garbling scheme of Bal~\etal~\cite{CCS:BalMalRos16}, 
a projective, output-projective and garbled-output random secure garbling scheme,
then
protocol $\Prfe$ with authenticated channels in the $\Fot$-hybrid model securely realizes $\Frfe^{P}$  with respect to static corruptions for any threshold $\delta$, 
as long as the \password space and notion of distance are such that for any \password $\pw$, it is easy to compute another \password $\pw'$ such that $d(\pw, \pw') > \delta$.\iftoggle{full}{\footnote{This is used in the argument of indistinguishability of Games~\printgame{YGCRFEbuildF} and \printgame{YGCRFEkeyshonestclose} in Appendix~\ref{sec:rfeproof}.}}{}
\end{theorem}
\begin{proof}[Sketch]
For every efficient adversary $\AdvA$, we describe a simulator $\SimYGCRFE$ such that no efficient environment can distinguish an execution with the real protocol $\Prfe$ and $\AdvA$ from an execution with the ideal functionality $\Frfe^{P}$ and $\SimYGCRFE$.
$\SimYGCRFE$ is described in \iftoggle{full}{Figure~\ref{fig:SimulatorYGCRFE}}{the full version of this paper}.
We prove indistinguishability in a series of hybrid steps.
First, we introduce the ideal functionality as a dummy node.
Next, we allow the functionality to choose the parties' keys, and we prove the indistinguishability of this step from the previous using the garbled output randomness property of our garbling scheme \iftoggle{full}{(Definition~\ref{def:garbledoutputrandomness}, Theorem~\ref{theorem:garbledoutputrandomness})}{}.
Next, we simulate an honest party's interaction with another honest party without using their \password, and prove the indistinguishability of this step from the previous using the obliviousness property of our garbling scheme. %(Definition~\ref{def:obliviousness}).
Finally, we simulate an honest party's interaction with a corrupted party without using the honest party's \password, and prove the indistinguishability of this step from the previous using the privacy property of our garbling scheme. %(Definition~\ref{def:privacy}).
\end{proof}
We give a more formal proof of Theorem~\ref{theorem:YGCRFE} in \appref{sec:rfeproof}.


% !TEX root = ../main.tex
% !TEX spellcheck = en-US

\subsubsection{From Split Randomized Fuzzy Equality to \FAKE}
\label{sec:ygcfaketransformation}

The Randomized Fuzzy Equality (RFE) functionality $\Frfe^{P}$ assumes authenticated channels, which an $\FAKE$ protocol cannot do.
In order to adapt RFE to our setting, we use the split functionality transformation defined by Barak~\etal~\cite{C:BCLPR05}.
Barak~\etal provide a generic transformation from protocols which require authenticated channels to protocols which do not.
In the ``transformed'' protocol, an adversary can engage in two separate instances of the protocol with the sender and receiver, and they will not realize that they are not talking to one another.
However, it does guarantee that the adversary cannot do anything beyond this attack.
In other words, it provides ``session authentication'', meaning that each party is guaranteed to carry out the entire protocol with the same partner, but not ``entity authentication'', meaning that the identity of the partner is not guaranteed.

Barak~\etal achieve this transformation in three steps.
First, the parties generate signing and verification keys, and send one another their verification keys.
Next, the parties sign the list of all keys they have received (which, in a two-party protocol, consists of only one key), sign that list, and send both list and signature to all other parties.
Finally, they verify all of the signatures they have received.
After this process --- called ``link initialization'' --- has been completed, the parties use those public keys they have exchanged to authenticate subsequent communication. 

We describe the Randomized Fuzzy Equality Split Functionality in Figure~\ref{fig:func-SRFE}.
It is simplified from Figure 1 in Barak~\etal~\cite{C:BCLPR05} because we only need to consider two parties and static corruptions.

\begin{figure}[tbp]
  \centering
  \begin{fboxenv}
    \begin{minipage}{0.95\textwidth}
      The functionality $\Fsrfe^{P}$ is parameterized by a security parameter~$\secparam$.
      It interacts with an adversary~$\Sx$ and two parties $\firstparty$ and $\secondparty$ via the following queries:\\[-1.8em]
      \begin{itemize}
      \item \textbf{Initialization}
        \begin{itemize}
        \item
        \textbf{Upon receiving a query $(\Init, \sid)$ from a party $\aparty \in \{\firstparty,\secondparty\}$}, 
        send $(\Init, \sid, \aparty)$ to the adversary $\Sx$.
        \item
        \textbf{Upon receiving a query $(\Init, \sid, \aparty, H, \sid_H)$ from the adversary $\Sx$}:
          \begin{itemize}
          \item
          Verify that $H \subseteq \{\firstparty, \secondparty\}$, that $\aparty \in H$, and that if a previous set $H'$ was recorded, either (1) $H \cap H'$ contains only corrupted parties and $\sid_H \neq \sid_{H'}$, or (2) $H = H'$ and $\sid_H = \sid_{H'}$.
          \item
          If verification fails, do nothing.
          \item 
          Otherwise, 
          record the pair $(H, \sid_H)$ (if it was not already recorded), 
          output $(\Init, \sid, \sid_H)$ to $\aparty$, and
          locally initialize a new instance of the original RFE functionality $\Frfe$ denoted $H\Frfe^{P}$, letting the adversary play the role of $\{\firstparty, \secondparty\} - H$ in $H\Frfe^{P}$.
          \end{itemize}
        \end{itemize}
      \item \textbf{RFE}
        \begin{itemize}
        \item
        \textbf{Upon receiving a query from a party $\aparty \in \{\firstparty, \secondparty\}$,}
        find the set $H$ such that $\aparty \in H$, and forward the query to $H\Frfe^{P}$.
        Otherwise, ignore the query.
        \item
        \textbf{Upon receiving a query from the adversary $\Sx$ on behalf of $\aparty$ corresponding to set $H$,}
        if $H\Frfe^{P}$ is initialized and $\aparty \not\in H$, then forward the query to $H\Frfe^{P}$.
        Otherwise, ignore the query.
        \end{itemize}
      \end{itemize}
    \end{minipage}
  \end{fboxenv}
  \caption{Functionality $\Fsrfe^{P}$}
  \label{fig:func-SRFE}
\end{figure}

It turns out that $\Fsrfe^{P}$ is enough to realize $\FFAKE^{P}$.
In fact, the protocol $\Prfe$ with the split functionality transformation directly realizes $\FFAKE^{P}$.
In \appref{sec:ygcfakeproof}, we prove that this is the case.


% !TEX root = ../main.tex
% !TEX spellcheck = en-US

\subsection{An Efficient Circuit $\func$ for Hamming Distance}
\label{sec:efficientf}

The Hamming distance of two \passwords $\pw, \pw' \in\F_{\alphabetsize}^{\pwlen}$ is equal to the number of locations at which the two \passwords have the same character. 
More formally,
\[d(\pw,\pw'):=|\left\{j\,|\,\pw[j]\neq\pw'[j], j\in[\pwlen]\right\}|.\]

We design $\func$ for Hamming distance as follows:
\begin{enumerate}
\item
First, $\func$ XORs corresponding (binary) \password characters, resulting in a list of bits indicating the (in)equality of those characters.
\item
Then, $\func$ feeds those bits into a threshold gate, which returns $1$ if at least $\pwlen - \delta$ of its inputs are $0$, and returns $0$ otherwise.
$\func$ returns the output of that threshold gate, which is $1$ if and only if at least $\pwlen - \delta$ \password characters match.
\end{enumerate}

This circuit, illustrated in Figure~\ref{fig:fakecircuit}, is very efficient to garble; it only requires $\pwlen$ ciphertexts.
Below, we briefly explain this garbling.
Our explanation assumes familiarity with YGC literature~\cite[and references therein]{YGCintro}.
Briefly, garbled gadget labels~\cite{CCS:BalMalRos16} enable the evaluation of modular addition gates for free (there is no need to include any information in the garbled circuit to enable this addition).
However, for a small modulus $m$, converting the output of that addition to a binary decision requires $m-1$ ciphertexts.
We utilize garbled gadgets with a modulus of $\pwlen + 1$ in our efficient garbling as follows:

\begin{enumerate}
\item
\label{item:eqcheck}
The input wire labels encode $0$ or $1$ modulo $\pwlen + 1$.
However, instead of having those input wire labels encode the characters of the two \passwords directly, they encode the outputs of the comparisons of corresponding characters.
If the $j$th character of $\aparty$'s \password is $0$, then $\aparty$ puts the $0$ label first; 
however, if the $j$th character of $\aparty$'s \password is $1$, then $\aparty$ flips the labels. 
Then, when $\otherparty$ is using oblivious transfer to retrieve the label corresponding to her $j$th \password character, she will retrieve the $0$ label if the two characters are equal, and the $1$ label otherwise.
(Note that this pre-processing on the garbler's side eliminates the need to send $\ginp_{0,0}$ and $\ginp_{1,1}$ in Figure~\ref{fig:YGCRFE}.)
\item
\label{item:threshold}
Compute a $\pwlen$-input threshold gate, as illustrated in Figure~6 of Yakoubov~\cite{YGCintro}. %Figure~\ref{fig:ninputthreshold}.
This gate returns $0$ if the sum of the inputs is above a certain threshold (that is, if at least $\pwlen - \delta$ \password characters differ), and $1$ otherwise.
This will require $\pwlen$ ciphertexts.
\end{enumerate}

Thus, a garbling of $\func$ consists of $\pwlen$ ciphertexts.
Since \FAKE requires two such garbled circuits (Figure~\ref{fig:YGCRFE}), $2\pwlen$ ciphertexts will be exchanged.

% !TEX root = ../main.tex
% !TEX spellcheck = en-US

\newcommand{\maxx}{12}
\newcommand{\maxy}{6}
\newcommand{\inputx}{0}
\newcommand{\xorx}{\maxx/3}
\newcommand{\thresholdx}{2*\maxx/3}
\newcommand{\outputx}{\maxx}
\newcommand{\inputygap}{\maxy/7}
\newcommand{\midy}{3.5*\inputygap}

\begin{figure}[tb]
\begin{center}
\begin{tikzpicture}[
	circuit logic US,
	tiny circuit symbols,
	every circuit symbol/.style={fill=white,draw, logic gate input sep=4mm},
	scale=0.70
]

\node (c1) at (\inputx,7*\inputygap) {$\firstpw^1$};
\node (s1) at (\inputx,6*\inputygap) {$\secondpw^1$};
\node (c2) at (\inputx,5*\inputygap) {$\firstpw^2$};
\node (s2) at (\inputx,4*\inputygap) {$\secondpw^2$};
\node (c3) at (\inputx,3*\inputygap) {$\firstpw^3$};
\node (s3) at (\inputx,2*\inputygap) {$\secondpw^3$};
\node (c4) at (\inputx,\inputygap) {$\firstpw^4$};
\node (s4) at (\inputx,0) {$\secondpw^4$};

\node (key) at (\outputx,\midy) {$\{0,1\}$};

\node [xor gate] at (\xorx,6.5*\inputygap) (xor1) {eq};
\node [xor gate] at (\xorx,4.5*\inputygap) (xor2) {eq};
\node [xor gate] at (\xorx,2.5*\inputygap) (xor3) {eq};
\node [xor gate] at (\xorx,.5*\inputygap) (xor4) {eq};
\node [and gate, inputs=nnnn] at (\thresholdx,\midy) (threshold1) {threshold};

\draw (c1) -- (xor1.input 1);
\draw (s1) -- (xor1.input 2);
\draw (c2) -- (xor2.input 1);
\draw (s2) -- (xor2.input 2);
\draw (c3) -- (xor3.input 1);
\draw (s3) -- (xor3.input 2);
\draw (c4) -- (xor4.input 1);
\draw (s4) -- (xor4.input 2);

\draw (xor1.output) -- (threshold1.input 1);
\draw (xor2.output) -- (threshold1.input 2);
\draw (xor3.output) -- (threshold1.input 3);
\draw (xor4.output) -- (threshold1.input 4);

\draw (threshold1.output) -- (key);

\end{tikzpicture}
\end{center}\vspace*{-2em}
\caption{The $\func$ circuit}
\label{fig:fakecircuit}
\end{figure}

\paragraph{Larger \Password Characters.}

If larger \password characters are used, then Step~\ref{item:eqcheck} above needs to change to check (in)equality of the larger characters instead of bits.
Step~\ref{item:threshold} will remain the same.
There are several ways to perform an (in)equality check on characters in $\F_{\alphabetsize}$ for $\alphabetsize \geq 2$:
\begin{enumerate}
\item 
Represent each character in terms of bits.
Step~\ref{item:eqcheck} will then consist of XORing corresponding bits, and taking an OR or the resulting XORs of each character to get negated equality.
This will take an additional $\pwlen \log(\alphabetsize)$ ciphertexts for every \password character.
\item
Use garbled gadget labels from the outset. 
We will require a larger OT ($1$-out-of-$\alphabetsize$ instead of $1$-out-of-$2$), but nothing else will change.
\end{enumerate}

