% !TEX root = ../main.tex
% !TEX spellcheck = en-US
In this section, we give a brief comparison of our \FAKE protocols.
First, in Figure~\ref{fig:comparison}, we describe the assumptions necessary for the two constructions, and the security parameters that they can achieve.

% \pad[inline]{I don't think the "Minimum Distance for Security $\gamma$" title is very easy to grasp. I would say "Minimum security gap $\gamma - \delta$", and adjust the values accordingly}

\newcolumntype{P}[1]{>{\centering\arraybackslash}p{#1}}

\begin{figure}[h!]
\centering
\begin{tabular}{|l||P{6cm}|P{2cm}|P{2cm}|}
\hline
& Assumptions & Threshold $\delta$ & Gap $\gamma-\delta$ \\ \hline \hline
$\PfakeRSS$ & UC-secure $\liPAKE$ & $<\pwlen/2$ & $\delta$ \\ \hline
$\PfakeYGC$ & (1) UC-secure OT (2) projective, output-projective and garbled-output random secure garbling scheme & Any & None \\ \hline
\end{tabular}
\caption{Assumptions, Distance Thresholds and Functionality/Security Gaps achieved by the two schemes.
$\PfakeRSS$ is the construction in Figure~\ref{fig:FAKERSS}.
$\PfakeYGC$ is the construction in Figure~\ref{fig:YGCRFE} with the split functionality transformation of Barak~\etal~\cite{C:BCLPR05}.}
\label{fig:comparison}
\end{figure}

Then, in Figure~\ref{fig:concretecomparison}, we describe the efficiency of the constructions when concrete primitives (OT / $\liPAKE$) are used to instantiate them.
$\PfakeRSS$ is instantiated as the construction in Figure~\ref{fig:FAKERSS} with the \liPAKE in Figure~\ref{fig:EKE2} and an \RSS.
% \pad{Since we do not break down the Shares and Reconstruct operations in the table, we should not detail more our choice of RSS.}
$\PfakeYGC$ is instantiated as the construction in Figure~\ref{fig:YGCRFE} with the UC-secure oblivious transfer protocol of Chou and Orlandi~\cite{LC:ChoOrl15}\iftoggle{full}{ described in Figure~\ref{fig:concreteOT}}{}, with the garbling scheme of Bal~\etal~\cite{CCS:BalMalRos16}, and with the split functionality transformation of Barak~\etal~\cite{C:BCLPR05}.
Though $\PfakeYGC$ can handle any efficiently computable notion of distance, Figure~\ref{fig:concretecomparison} assumes that both constructions use Hamming distance (and that, specifically, $\PfakeYGC$ uses the circuit described in Figure~\ref{fig:fakecircuit}).
We describe efficiency in terms of sub-operations (per-party, not in aggregate).

Note that these concrete primitives each have their own set of required assumptions.
Specifically, the \liPAKE in Figure~\ref{fig:EKE2} requires a random oracle (RO\iftoggle{full}{, described in Figure~\ref{functionality:RO}}{}), ideal cipher (IC\iftoggle{full}{, described in Figure~\ref{functionality:IC}}{}) and common reference string (CRS\iftoggle{full}{, described in Figure~\ref{functionality:CRS}}{}). 
The oblivious transfer protocol \iftoggle{full}{in Figure~\ref{fig:concreteOT}}{of Chou and Orlandi~\cite{LC:ChoOrl15}} requires a random oracle.
The garbling scheme of Bal~\etal~\cite{CCS:BalMalRos16} requires a mixed modulus circular correlation robust hash function, which is a weakening of the random oracle assumption.

For $\PfakeRSS$, the factor of $\pwlen$ arises from the $\pwlen$ times EKE2 is executed. 
For $\PfakeYGC$, the factor of $\pwlen$ comes from the garbled circuit.
Additionally, in $\PfakeYGC$, three communication flows come from OT.
The last of these is combined with sending the garbled circuits.
%one is required to send the garbled circuits, and 
Two additional flows of communication come from the split functionality transformation.
The need for signatures also arises from the split functionality transformation.

\newcolumntype{R}[2]{
    >{\adjustbox{angle=#1,lap=\width-(#2)}\bgroup}
    l
    <{\egroup}
}
\newcommand*\rotsixty{\multicolumn{1}{|R{60}{0em}|}}
\newcommand*\rotseventyfive{\multicolumn{1}{|R{75}{0em}|}}
\newcommand*\roteighty{\multicolumn{1}{|R{80}{0em}|}}
\newcommand*\rotninety{\multicolumn{1}{|R{90}{0em}|}}

\begin{figure}[h!]
\centering
\begin{tabular}{|l|l||p{1cm}|p{1.3cm}|c|c|p{.6cm}|p{.6cm}|p{.6cm}|p{.6cm}|p{.6cm}|p{.6cm}|p{.6cm}|}
\hline
& & 
\roteighty{Output Key Format} & 
\roteighty{$\#$ \begin{minipage}[t]{2.5cm}(Bidirectional) Communication Flows\end{minipage}} &
\roteighty{$\#$ Exponentiations} &
\roteighty{$\#$ Hashes} &
\roteighty{$\#$ Encryptions} &
\roteighty{$\#$ Decryptions} &
\roteighty{$\#$ Share} &
\roteighty{$\#$ Reconstruct} &
\roteighty{$\#$ SigKeyGens} &
\roteighty{$\#$ Signs} &
\roteighty{$\#$ SigVerifies} 
\\ \hline \hline
\multirow{2}{*}{$\PfakeRSS$} & sender & \multirow{2}{*}{$\F_q$} & \multirow{2}{*}{$2$} & \multirow{2}{*}{$2\pwlen$} & \multirow{2}{*}{$\pwlen$} & \multirow{2}{*}{$\pwlen$} & \multirow{2}{*}{$\pwlen$} & $1$ & $0$ & $1$ & $1$ & $0$ \\  \hhline{~-~~~~~~-----}
& receiver & & & & & & & $0$ & $1$ & $0$ & $0$ & $1$ \\  \hline
\multicolumn{2}{|c||}{$\PfakeYGC$} & $\{0,1\}^{\secparam}$ & 5 & $3\pwlen + 2$ & $4\pwlen + 7$ & $2\pwlen$ & $\pwlen$ & $-$ & $-$ & $1$ & $5$ & $5$ \\ \hline
\end{tabular}
\caption{Efficiency (in Terms of Sub-Operations) of the Two Constructions.
Here, by ``bidirectional communication flow'' we mean two flows, one in each direction, which do not depend on one another.
$\PfakeRSS$ is the construction in Figure~\ref{fig:FAKERSS} instantiated with the \liPAKE in Figure~\ref{fig:EKE2}. 
The first $\PfakeRSS$ row describes the sender's efficiency; the second row describes the receiver's efficiency.
$\PfakeYGC$ is the construction in Figure~\ref{fig:YGCRFE} instantiated with the UC-secure oblivious transfer protocol of Chou and Orlandi~\cite{LC:ChoOrl15}\iftoggle{full}{ described in Figure~\ref{fig:concreteOT}}{}, the garbling scheme of Bal~\etal~\cite{CCS:BalMalRos16}, and with the split functionality transformation of Barak~\etal~\cite{C:BCLPR05}.
$\PfakeYGC$ is described in a single row, since it is a symmetric protocol.}
\label{fig:concretecomparison}
\end{figure}

\expl{In the YGC column of the table...
The $3\pwlen + 2$ exponentiations come from OT.
$3\pwlen$ hashes come from OT, $\pwlen$ hashes come from garbling, and $1$ hash comes from evaluating.
One more hash comes from the final key derivation step.
In the split functionality framework, $5$ more hashes come from using the hash-and-sign paradigm.
$2\pwlen$ symmetric encryptions come from OT.
%Another $\pwlen$ symmetric encryptions come from garbling.
$\pwlen$ symmetric decryptions come from OT.
%Another symmetric decryption come from evaluating.
3 communication flows come from OT (the last round of which can be combined with circuit transmission).
2 communication flows come from the split functionality transformation.

In the \WRSS column: everything from EKE2 is taken times $\pwlen$ executions. So EKE2 leads to $2\pwlen$ exp, $\pwlen$ hashes, encryptions and decryptions. 
The rest of the protocol is one of each of the following: Reconstruct, Share, Sign, Vfy. Also $2$ group operations.
}

\paragraph{Efficiency Optimizations to $\PfakeYGC$}

We can make several small efficiency improvements to the $\PfakeYGC$ construction which are not reflected in Figure~\ref{fig:concretecomparison}.
%(They are not reflected in Figure~\ref{fig:concretecomparison}, because they are not a part of the scheme the security of which we prove.)
%First, the last flow of the OT protocol can be combined with the sending of the garbled circuits, reducing the number of bidirectional communication flows from $6$ to $5$.
First, instead of using the split functionality transformation of Barak~\etal~\cite{C:BCLPR05}, we can use the split split functionality of Camenisch~\etal~\cite{C:CCGS10}.
It uses a split key exchange functionality to establish symmetric keys, and then uses those to symmetrically encrypt and authenticate each flow.
While this does not save any communication rounds, it does reduce the number of public key operations needed.
%Finally, 
Second, if the \passwords are more than $\secparam$ bits long (where $\secparam$ is the security parameter), OT extensions that are secure against malicious adversaries~\cite{EC:AHMR15} can be used.
If the \passwords are fewer than $\secparam$ bits long, then nothing is to be gained from using OT extensions, since OT extensions require $\secparam$ ``base OTs''. 
However, if the \passwords are longer --- say, if they are some biometric measurement that is thousands of bits long --- then OT extensions would save on the number of public key operations, at the cost of an extra round of communication. 

\sophia[inline]{Leo wants to incorporate OT extensions into the table.}
\sophia[inline]{Is it possible to leverage OT extensions from \fPAKE session to session?}

%\begin{figure}[h!]
%\begin{tabular}{|c|c|c|c|c|c|c|c|c|c|}
%\hline
%Construction & Exponentiate & Hash & Encrypt & Decrypt & Encode & Decode & SigKeyGen & Sign & Verify \\ \hline \hline
%$\FAKE_2$ & $3\pwlen$ & $\pwlen$ & $\pwlen$ & $\pwlen$ & $2$ & $1$ & $1$ & $1$ & $1$ \\ \hline
%YGC with split functionality & $3\pwlen + 2$ & $4\pwlen + 7$ & $2\pwlen$ & $\pwlen$ & $-$ & $-$ & $1$ & $5$ & $5$ \\ \hline
%\end{tabular}
%
%\begin{tabular}{|c|c|c|c|c|}
%\hline
%Construction & Assumptions & Rounds of Communication & Gap & $\delta$ \\ \hline \hline
%$\FAKE_2$ & RO, IC, CRS & $2$ & ? & ? \\ \hline
%YGC with split functionality & UC OT, MMCCR hash function & $6$ & None & Any \\ \hline
%\end{tabular}
%\label{fig:comparison}
%\end{figure}
