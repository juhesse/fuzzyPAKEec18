% !TEX root = ../main.tex
% !TEX spellcheck = en-US

\heading{Common Reference String}
The Common Reference String (CRS) functionality was already defined in~\cite{AC:Canetti07}.
We recall it in Figure~\ref{functionality:CRS} for completeness.
Note that we do not let \Fcrs check whether a party is allowed to obtain the CRS --- it is assumed public.
\begin{figure}[htbp]
\noindent
\fbox{\begin{minipage}{\linewidth-9pt}
The functionality $\Fcrs^{\mathcal{D}}$ is parameterized with a distribution $\mathcal{D}$ and proceeds as follows:
\begin{itemize}
 \item Upon receiving $(\sid,\web{crs})$:
 \begin{itemize}
 \item
 If there is no value $r$ recorded, then choose and record a value $r\getsr\mathcal{D}$.
 \item
 Reply with $(\sid,r)$.
 \end{itemize}
\end{itemize}
\end{minipage}}
\caption{Functionality $\Fcrs$} \label{functionality:CRS}
\end{figure}

\heading{Random Oracles}
The Random Oracle (RO) functionality was already defined by Hofheinz and M{\"u}ller-Quade in~\cite{TCC:HofMul04}.
We recall it in Figure~\ref{functionality:RO} for completeness.
It is clear that the random oracle model UC-emulates this functionality.
\begin{figure}[htbp]
\noindent
\fbox{\begin{minipage}{\linewidth-9pt}
The functionality $\Fro$ proceeds as follows, running on security
parameter~$k$, with a set of (dummy) parties $\Party_1,\ldots,\Party_n$ and an adversary~\Sim:
\begin{itemize}
\item
$\Fro$ keeps a list~$L$ (which is initially empty) of pairs of
bit strings.
\item
Upon receiving a value $(\sid,m)$  (with $m\in\{0,1\}^*$) from some
party~$\Party_i$ or from~\Sim, do:
    \begin{itemize}
    \item
    If there is a pair $(m,\tilde{h})$ for some $\tilde{h}\in\{0,1\}^k$
    in the list~$L$, set $h := \tilde{h}$.
    \item
    If there is no such pair, choose uniformly $h\in\{0,1\}^k$ and store
    the pair $(m,h)\in L$.
    \end{itemize}
Once $h$ is set, reply to the activating machine (i.e., either $\Party_i$
or \Sim) with $(\sid,h)$.
\end{itemize}
\end{minipage}}
\caption{Functionality $\Fro$} \label{functionality:RO}
\end{figure}

\heading{Ideal Cipher}
An ideal cipher~\cite{EC:BelPoiRog00} is a block cipher that takes a plaintext or a ciphertext as input.
We describe the ideal cipher functionality \Fic in Figure~\ref{functionality:IC}, in the same vein as the above random oracle functionality.
It is clear that the ideal cipher model UC-emulates this functionality.
Note that this functionality characterizes a perfectly random permutation for each key by ensuring injectivity for each query simulation: to this aim, it uses a list $L$ and projections $M_\sk$ and $C_\sk$, that are global, independently of the $\sid$.
\begin{figure}[htbp]
\noindent
\fbox{\begin{minipage}{\linewidth-10pt}
The functionality \Fic takes as input the security
parameter~$k$, and interacts with an adversary~\Sim and with a set
of (dummy) parties $\Party_1,\ldots,\Party_n$ by means of these queries:
\begin{itemize}
\item
    \Fic keeps a (initially empty) list~$L$ containing
$3-$tuples of bit strings and two (initially empty) sets $C_{\sk}$ and $M_{\sk}$ for every $\sk$.
(The sets are not created until $\sk$ is first used, thus avoiding the need to instantiate exponentially many sets.)
\item
    \textbf{\mathversion{bold}Upon receiving a query $(\sid,\Encrypt,\sk,m)$
    (with $m\in\{0,1\}^k$) from some party~$\Party_i$ or~\Sim, 
do:\mathversion{normal}}
    \begin{itemize}
    \item
        If there is a $3-$tuple $(\sk,m,\tilde{c})$ for some
$\tilde{c}\in\{0,1\}^k$ in the list~$L$, set $c := \tilde{c}$.
    \item
        If there is no such record, choose uniformly $c \in \{0,1\}^k\backslash C_{\sk}$
        which is the set consisting of ciphertexts not already used with $\sk$.
        Next, it stores the $3-$tuple $(\sk,m,c)\in L$ and sets both
        $M_{\sk}\gets M_{\sk}\cup \{m\}$ and $C_{\sk}\gets C_{\sk}\cup \{c\}$.
    \end{itemize}
Once $c$ is set, reply to the activating machine with $(\sid,c)$.
\item
    \textbf{\mathversion{bold}Upon receiving a query $(\sid,\Decrypt,\sk,c)$  (with
$c\in\{0,1\}^k$) from some party~$\Party_i$ or~\Sim, do:\mathversion{normal}}
    \begin{itemize}
    \item
        If there is a $3-$tuple $(\sk,\tilde{m},c)$ for some
$\tilde{m}\in\{0,1\}^k$ in~$L$, set $m := \tilde{m}$.
    \item
        If there is no such record, choose uniformly $m \in \{0,1\}^k\backslash M_{\sk}$
        which is the set consisting of plaintexts not already used with
        $\sk$. Next, it stores the $3-$tuple $(\sk,m,c)\in L$ 
and sets both
        $M_{\sk}\gets M_{\sk}\cup \{m\}$ and $C_{\sk}\gets C_{\sk}\cup \{c\}$.
    \end{itemize}
Once $m$ is set, reply to the activating machine with $(\sid,m)$.

\end{itemize}
\end{minipage}}
\caption{Functionality \Fic} \label{functionality:IC}
\end{figure}

\heading{Oblivious Transfer}
The Oblivious Transfer (OT) functionality was defined by Canetti \etal~\cite{STOC:CLOS02}.
We recall it in Figure~\ref{fig:func-OT}.

\begin{figure}[htbp]
  \centering
  \begin{fboxenv}
    \begin{minipage}{0.95\textwidth}
      The functionality $\Fot$ is parameterized by a security parameter~$\secparam$.
      It interacts with an adversary~$\Sx$ and the players $\sender$ (the sender) and $\receiver$ (the receiver) via the following queries:
      \begin{itemize}
      \item
        \textbf{Upon receiving a query
        \mathversion{bold}$(\Send,\sid,x_0,x_1)$ from $\sender$\mathversion{normal}}, where $x_0,x_1\in\bits^{\secparam}$, record the tuple $(x_0,x_1)$.
      \item
        \textbf{Upon receiving a query
        \mathversion{bold}$(\Receive,\sid,i)$ from $\receiver$\mathversion{normal}:} \\
        If there is a record~$(x_0,x_1)$, then send $(\sid,x_i)$ to $\receiver$ and $\sid$ to $\server$, and halt.
        Otherwise, ignore the query.
      \end{itemize}
    \end{minipage}
  \end{fboxenv}
  \caption{Functionality $\Fot$}\label{fig:func-OT}
\end{figure}

\heading{Password-Authenticated Key Exchange}
The initial \PAKE functionality $\FpwKE$ has been defined by Canetti \etal~\cite{EC:CHKLM05}. We recall it in Figure~\ref{fig:func-PAKE}. We stress that this functionality immediately leaks the result of the \TestPwd-query, which models explicit authentication; when the adversary tries a password, it learns whether the guess was correct or not.

\begin{figure}[htbp]
  \centering
  \begin{fboxenv}
    \begin{minipage}{0.95\textwidth}
      The functionality $\FpwKE$ is parameterized by a security parameter~$k$.
      It interacts with an adversary~$\Sx$ and a set of (dummy) parties $\Party_1,\ldots,\Party_n$ via the following queries:
      \begin{itemize}
      \item
        \textbf{Upon receiving a query
        \mathversion{bold}$(\NewSession,\sid,\Party_i,\Party_j,\pw,\role)$ from party $\Party_i$\mathversion{normal}:}
        \begin{itemize}
          \item Send $(\NewSession,\sid,\Party_i,\Party_j,\role)$ to \Sim;
          \item If one of the following is true, record~$(\Party_i,\Party_j,\pw)$ and mark this record \web{fresh}:
          \begin{itemize}
             \item This is the first \NewSession query %and $\role = \rolesender$
             \item This is the second \NewSession query %, $\role = \rolereceiver$ 
             and there is a record~$(\Party_j,\Party_i,\pw')$
          \end{itemize}
%          \item If this is the first \NewSession query,
%          or if this is the second \NewSession query and there is a record~$(\Party_j,\Party_i,\pw')$,
%          then record~$(\Party_i,\Party_j,\pw)$ and mark this record \web{fresh}.
        \end{itemize}
      \item
        \textbf{Upon receiving a query
        \mathversion{bold}$(\TestPwd,\sid,\Party_i,\pw')$ from \Sim\mathversion{normal}:} \\
        If there is a \web{fresh} record~$(\Party_i,\Party_j,\pw)$, then do: 
        \begin{itemize}
          \item If $\pw=\pw'$, mark the record \web{compromised}, and reply to $\Sx$ with \emph{correct guess};
          \item If $\pw\neq\pw'$, mark the record \web{interrupted}, and reply to $\Sx$ with \emph{wrong guess}.
        \end{itemize}
      \item
        \textbf{Upon receiving a query
          \mathversion{bold}$(\NewKey,\sid,\Party_i,\sk)$ from the \Sim, where $|\sk|=k$:\mathversion{normal} } \\
        If there is a record of the form~$(\Party_i,\Party_j,\pw)$, and this is the first \NewKey query for~$\Party_i$, then:
        \begin{itemize}
	  \item If this record is \web{compromised}, or either $\Party_i$ or $\Party_j$ is corrupted, then output~$(\sid,\sk)$ to player~$\Party_i$;
          \item If this record is \web{fresh}, there is a record~$(\Party_j,\Party_i,\pw')$ with $\pw=\pw'$, a key~$\sk'$ was sent to~$\Party_j$,
          and $(\Party_j,\Party_i,\pw)$ was \web{fresh} at the time, then output~$(\sid,\sk')$ to~$\Party_i$;
          \item In any other case, pick a new random key~$\sk'$ of length~$k$ and send~$(\sid,\sk')$ to~$\Party_i$.
        \end{itemize}
        Either way, mark the record $(\Party_i,\Party_j,\pw)$ as \web{completed}.
      \end{itemize}
    \end{minipage}
  \end{fboxenv}
  \caption{Functionality $\FpwKE$}\label{fig:func-PAKE}
\end{figure}

In our paper, we start from this functionality to derive the basic functionality \FAKE, in Section~\ref{sec:model}, after a few changes:
\begin{itemize}
\item we consider only two parties --- $\firstparty$ and $\secondparty$ ---, which is enough since universal composability takes care of ensuring that a two-party functionality remains secure in a multi-party world;
\item we do not allow the adversary to set $\aparty$'s key if $\otherparty$ is corrupted but did not guess $\aparty$'s password.
We make this change in order to protect an honest $\aparty$ from, for instance, revealing sensitive information to an adversary who did not successfully guess her password, but did corrupt her partner.
\end{itemize}